Sebuah bilangan asli $n$ tertulis di papan. Pada setiap langkah, Neneng dan Asep mengubah angka di papan dengan peraturan sebagai berikut: Misalkan angka di papan adalah $X$. Awalnya, Neneng memilih tanda naik atau turun. Kemudian, Asep memilih suatu pembagi positif $d$ dari $X$, dan mengganti $X$ dengan $X+d$ jika Neneng memilih naik, atau $X-d$ jika Neneng memilih turun. Asep menang jika bilangan di papan merupakan suatu bilangan kuadrat sempurna tak nol, dan kalah jika di suatu saat ia menuliskan angka nol. Buktikan jika $n \ge 14$, Asep dapat menang dalam paling banyak $(n-5)/4$ langkah.