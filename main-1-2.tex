\documentclass[12pt]{scrartcl}
\usepackage[sexy]{evan}
\usepackage{graphicx,amsmath,amssymb,amsthm,amsfonts,babel}
\usepackage{tikz, tkz-euclide}
\usepackage{lipsum}
\usepackage{setspace}
\graphicspath{ {./} }
\usetikzlibrary{calc,through,intersections}
\usepackage[paperwidth=16cm, paperheight=16cm,margin=1cm]{geometry}

\colorlet{EvanRed}{Red!50!Purple}

\newcommand{\siku}[4][.5cm]
	{
	\coordinate (tempa) at ($(#3)!#1!(#2)$);
	\coordinate (tempb) at ($(#3)!#1!(#4)$);
	\coordinate (tempc) at ($(tempa)!0.5!(tempb)$);%midpoint
	\draw[black] (tempa) -- ($(#3)!2!(tempc)$) -- (tempb);
	}
	\usetikzlibrary{calc,positioning,intersections}

\setstretch{1.5}

\usepackage{etoolbox}
\newcommand{\zerodisplayskips}{%
  \setlength{\abovedisplayskip}{5pt}%
  \setlength{\belowdisplayskip}{5pt}%
  \setlength{\abovedisplayshortskip}{5pt}%
  \setlength{\belowdisplayshortskip}{5pt}}
\appto{\normalsize}{\zerodisplayskips}
\appto{\small}{\zerodisplayskips}
\appto{\footnotesize}{\zerodisplayskips}
\setlength\parindent{10pt}

\title{Solusi OSN Matematika SMA 2023 part 1}
\author{Nomor 1 dan 2 hari pertama}
\date{}

\begin{document}

\maketitle
\newpage
\section{Soal}
\subsection{Nomor 1 Hari Pertama}
Diberikan segitiga lancip $ABC$ dengan sisi terpanjang $BC$. Titik $D$,$E$ berturut-turut pada $AC$, $AB$ sehingga $BA=BD$ dan $CA=CE$. Titik $A'$ merupakan pencerminan $A$ terhadap garis $BC$. Buktikan bahwa lingkaran luar segitiga $ABC$ dan $A'DE$ mempunyai panjang jari-jari yang sama.
\subsection{Nomor 2 Hari Pertama}
Tentukan semua fungsi $f: \RR \to \RR$ sehingga untuk setiap bilangan real $x$, $y$ berlaku
    $$f(f(x)+y)=\floor{x+f(f(y))}.$$

\textbf{Catatan:} $\floor{x}$ menyatakan bilangan bulat terbesar yang kurang dari atau sama dengan $x$.

\section{Solusi}
\subsection{Nomor 1 Hari Pertama}
Diberikan segitiga lancip $ABC$ dengan sisi terpanjang $BC$. Titik $D$,$E$ berturut-turut pada $AC$, $AB$ sehingga $BA=BD$ dan $CA=CE$. Titik $A'$ merupakan pencerminan $A$ terhadap garis $BC$. Buktikan bahwa lingkaran luar segitiga $ABC$ dan $A'DE$ mempunyai panjang jari-jari yang sama.
\newline
\textit{(Soal dipropose oleh Reza Wahyu Kumara (Perunggu IMO 2014))}
\begin{motivasi}
    Dari syarat pencerminan $A$ ke $A'$ akan terbentuk segitiga yang kongruen dengan $ABC$ yaitu $A'BC$. Dari sini karena letak $A'DE$ "lebih dekat" dengan $A'BC$ maka saya iseng ingin buktikan kedua lingkaran tersebut ternyata punya satu lingkaran luar yang sama (dan ternyata benar). 
\end{motivasi}
\begin{solusi}[oleh Azzam L. H.]
    Notasikan $\dangle$ sebagai $\textit{directed angle} \mod 180^\circ$.
    Perhatikan karena $A'$ adalah refleksi dari $A$ terhadap $BC$, maka $AC=CA'$ dan $AB=BA'$ sehingga $\triangle ABC \cong \triangle A'BC$. Oleh karena itu, panjang jari-jari lingkaran luar segitiga $ABC$ dan $A'BC$ sama. Dari sini cukup dibuktikan bahwa ternyata lingkaran $(A'BC)$ dan $(A'DE)$ adalah lingkaran yang sama atau cukup dibuktikan bahwa $A',B,E,D,C$ berada di satu lingkaran yang sama.
    \begin{center}
        \begin{asy}
            import geometry;
            import olympiad;
            size(170);
            pair A = (0,1.75);
            pair B = (2,0);
            pair C = (-1,0);
            real ab = length(A-B);
            pair AD[] = intersectionpoints(line(A,C),circle(B,ab));
            pair D = AD[0];
            real ac = length(A-C);
            pair AE[] = intersectionpoints(line(A,B),circle(C,ac));
            pair E = AE[1];
            pair A1 = reflect(B,C)*A;
            circle abc = circumcircle(triangle(A,B,C));
            circle dea1 = circumcircle(triangle(A1,D,E));
            dot("$A$", A, N);
            dot("$B$", B, E);
            dot("$C$", C, W);
            dot("$D$", D, NW);
            dot("$E$", E, NE);
            dot("$A'$", A1, SW);
            draw(B--C--A1--B);
            draw(A--B--D, blue);
            draw(A--C--E, red);
            draw(abc);
            draw(dea1, dashed);
        \end{asy}
    \end{center}
    Perhatikan bahwa karena $AB=BD$ dan $AC=CE$ maka 
    \begin{align*}
        \dangle BDC = \dangle BDA = \dangle DAE = \dangle AEC = \dangle BEC
    \end{align*}
    yang menunjukkan $B,D,E,C$ berada di satu lingkaran. Di sisi lain 
    \begin{align*}
        \dangle BEC = \dangle AEC = \dangle CAB = \dangle BA'C
    \end{align*}
    yang menunjukkan $A'B,E,C$ berada di satu lingkaran. Dari kedua fakta tersebut, dapat disimpulkan bahwa $A',B,E,D,C$ berada di satu lingkaran yang sama, sesuai yang ingin dibuktikan.
\end{solusi}
\subsection{Nomor 2 Hari Pertama}
Tentukan semua fungsi $f: \RR \to \RR$ sehingga untuk setiap bilangan real $x$, $y$ berlaku
    $$f(f(x)+y)=\floor{x+f(f(y))}.$$

\textbf{Catatan:} $\floor{x}$ menyatakan bilangan bulat terbesar yang kurang dari atau sama dengan $x$.
\newline
\textit{(Soal di\textit{propose} oleh Valentio Iverson (Absolute Winner, Emas OSN 2019))}
\begin{motivasi}
    Jika dilihat pada soal, ekspresi $f(\dots) = \floor{\dots}$ menandakan bahwa $f$ bernilai bulat untuk semua $y$, oleh karena itu aman dikatakan bahwa ini adalah fungsi yang bernilai bulat. Lalu, saya sendiri coba "memaksa" agar $f(0)$ atau ekspresi yang mirip ditemukan agar mungkin bisa mendapatkan ekspresi yang lebih bagus lagi. Selain itu, dari ekspresi di soal saya mempunyai tujuan untuk membuktikan bahwa $f(\dots) = \floor{\dots} + \text{sesuatu}$.
\end{motivasi}
\begin{solusi}[oleh Azzam L. H.]
    Notasikan $P(x,y)$ sebagai asersi $x$ dan $y$ pada persamaan di soal. Pertama jika diasersi $P(x,-f(x)+y)$ untuk sebarang $x,y \in \RR$ maka 
    \begin{align*}
        f(y) = \floor{x + f(f(-f(x)+y))}
    \end{align*}
    yang menunjukkan bahwa $f: \RR \to \ZZ$ atau hasil pemetaan $f$ harus bulat.
    Selanjutnya, karena $f(f(0))$ bulat, dengan asersi $P(x,0)$ untuk sembarang $x \in \RR$ akan didapat
    \begin{align*}
        f(f(x)) = \floor{x+f(f(0))} = \floor{x}+f(f(0))
    \end{align*}
    oleh karena itu, jika diasersi $P(x,-f(x))$ untuk sembarang $x \in \RR$ dengan persamaan sebelumnya (ganti $x$ dengan $-f(x)$) dan karena hasil $f$ bulat, maka akan didapat
    \begin{align*}
        f(0) &= \floor{x+f(f(-f(x)))}\\
        f(0) &= \floor{x} + f(f(-f(x)))\\
        f(0) &= \floor{x} + \floor{-f(x)} + f(f(0))\\
        f(0) &= \floor{x} -f(x) + f(f(0))\\
        f(x) &= \floor{x} + f(f(0))-f(0)
    \end{align*}
    Karena $f(0)$ konstan dan bulat, maka $t = f(f(0))-f(0)$ juga bernilai konstan dan bulat sehingga $f(x) = \floor{x}+t$. Jika dicek ke soal,
    \begin{align*}
        f(f(x)+y)&=f(\floor{x}+t+y)\\
                &=\floor{\floor{x}+t+y}+t\\
                &=\floor{\floor{x}}+\floor{y}+t+t\\
                &= \floor{\floor{x}} + \floor{\floor{y}+t+t}\\
                &= \floor{\floor{x}+\floor{\floor{y}+t+t}}\\
                &= \floor{x+f(\floor{y}+t)}\\
                &= \floor{x+f(f(y))}
    \end{align*}
    ternyata $\boxed{f(x) = \floor{x} + t}$ untuk sembarang $x \in \RR$ dan konstanta $t \in \ZZ$ adalah fungsi yang memenuhi soal. 
    
\end{solusi}

\end{document}
