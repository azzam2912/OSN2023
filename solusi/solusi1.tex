Diberikan segitiga lancip $ABC$ dengan sisi terpanjang $BC$. Titik $D$,$E$ berturut-turut pada $AC$, $AB$ sehingga $BA=BD$ dan $CA=CE$. Titik $A'$ merupakan pencerminan $A$ terhadap garis $BC$. Buktikan bahwa lingkaran luar segitiga $ABC$ dan $A'DE$ mempunyai panjang jari-jari yang sama.
\newline
\textit{(Soal dipropose oleh Reza Wahyu Kumara (Perunggu IMO 2014))}
\begin{motivasi}
    Dari syarat pencerminan $A$ ke $A'$ akan terbentuk segitiga yang kongruen dengan $ABC$ yaitu $A'BC$. Dari sini karena letak $A'DE$ "lebih dekat" dengan $A'BC$ maka saya iseng ingin buktikan kedua lingkaran tersebut ternyata punya satu lingkaran luar yang sama (dan ternyata benar). 
\end{motivasi}
\begin{solusi}[oleh Azzam L. H.]
    Notasikan $\dangle$ sebagai $\textit{directed angle} \mod 180^\circ$.
    Perhatikan karena $A'$ adalah refleksi dari $A$ terhadap $BC$, maka $AC=CA'$ dan $AB=BA'$ sehingga $\triangle ABC \cong \triangle A'BC$. Oleh karena itu, panjang jari-jari lingkaran luar segitiga $ABC$ dan $A'BC$ sama. Dari sini cukup dibuktikan bahwa ternyata lingkaran $(A'BC)$ dan $(A'DE)$ adalah lingkaran yang sama atau cukup dibuktikan bahwa $A',B,E,D,C$ berada di satu lingkaran yang sama.
    \begin{center}
        \begin{asy}
            import geometry;
            import olympiad;
            size(170);
            pair A = (0,1.75);
            pair B = (2,0);
            pair C = (-1,0);
            real ab = length(A-B);
            pair AD[] = intersectionpoints(line(A,C),circle(B,ab));
            pair D = AD[0];
            real ac = length(A-C);
            pair AE[] = intersectionpoints(line(A,B),circle(C,ac));
            pair E = AE[1];
            pair A1 = reflect(B,C)*A;
            circle abc = circumcircle(triangle(A,B,C));
            circle dea1 = circumcircle(triangle(A1,D,E));
            dot("$A$", A, N);
            dot("$B$", B, E);
            dot("$C$", C, W);
            dot("$D$", D, NW);
            dot("$E$", E, NE);
            dot("$A'$", A1, SW);
            draw(B--C--A1--B);
            draw(A--B--D, blue);
            draw(A--C--E, red);
            draw(abc);
            draw(dea1, dashed);
        \end{asy}
    \end{center}
    Perhatikan bahwa karena $AB=BD$ dan $AC=CE$ maka 
    \begin{align*}
        \dangle BDC = \dangle BDA = \dangle DAE = \dangle AEC = \dangle BEC
    \end{align*}
    yang menunjukkan $B,D,E,C$ berada di satu lingkaran. Di sisi lain 
    \begin{align*}
        \dangle BEC = \dangle AEC = \dangle CAB = \dangle BA'C
    \end{align*}
    yang menunjukkan $A'B,E,C$ berada di satu lingkaran. Dari kedua fakta tersebut, dapat disimpulkan bahwa $A',B,E,D,C$ berada di satu lingkaran yang sama, sesuai yang ingin dibuktikan.
\end{solusi}