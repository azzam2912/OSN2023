Sebuah bilangan asli $n$ tertulis di papan. Pada setiap langkah, Neneng dan Asep mengubah angka di papan dengan peraturan sebagai berikut: Misalkan angka di papan adalah $X$. Awalnya, Neneng memilih tanda naik atau turun. Kemudian, Asep memilih suatu pembagi positif $d$ dari $X$, dan mengganti $X$ dengan $X+d$ jika Neneng memilih naik, atau $X-d$ jika Neneng memilih turun. Asep menang jika bilangan di papan merupakan suatu bilangan kuadrat sempurna tak nol, dan kalah jika di suatu saat ia menuliskan angka nol. Buktikan jika $n \ge 14$, Asep dapat menang dalam paling banyak $(n-5)/4$ langkah.
\newline
\textit{(Soal di\textit{propose} oleh Muhammad Afifurrahman (Emas OSN 2015))}
\begin{motivasi}(oleh Muhammad Afifurrahman)
    \begin{itemize}
        \item Motivasi untuk solusi ini berasal dari $f(n(n+1))=1$ (karena Asep dapat berpindah ke $n^2$ atau $(n+1)^2$ dalam kasus ini).
\item \textit{case work}nya tidak terlalu berat jika sudah memiliki klaim tersebut.
\item Dengan menguji semua kemungkinan, seseorang dapat membuktikan bahwa $f(17)=3$ yang menghasilkan sebuah kasus ketaksamaan.
\item Tentu saja, batasan klaim tersebut tidak optimal. Namun, klaim tersebut sudah cukup untuk membuktikan pernyataan di soal.
\item Selain itu, batasan masalah ini tidak optimal untuk nilai $n$ yang besar; khususnya, dengan memodifikasi (dan mempertimbangkan lebih banyak kasus modulo...), ada $C(k)$ sedemikian sehingga $f(n)\leq (n-C(k))/k$ untuk semua nilai $n$ yang cukup besar.
\item Lebih lanjut, teman saya menunjukkan bahwa $f(n) \leq \lceil \log_2 n \rceil$ untuk semua nilai $n$ yang cukup besar, dengan memperhatikan bahwa $f(2^{k})=1$ ketika $k$ ganjil. 
    \end{itemize}
\end{motivasi}
\begin{solusi}(oleh Muhammad Afifurrahman)
    Misalkan $f(n)$ menotasikan langkah terkecil yang dibutuhkan Asep untuk menang. 

    \begin{claim*}
        $f(xy)\leq |x-y|$. 
        \begin{bukti}
            \item Jika $x=y$, Asep sudah selesai. Andaikan $x \neq y$ , WLOG $x>y$, Asep dapat melanjutkan seperti berikut ini:
            \item Jika Neneng memilih naik, Asep dapat mengganti $xy$ dengan $x(y+1)$.
            \item Jika Neneng memilih turun, Asep dapat mengganti $xy$ dengan $(x-1)y$. Langkah ini valid karena $x\geq 2$.
            Dalam setiap langkah, Asep akan mengurangi selisih antara kedua bilangan tersebut sebanyak tepat 1. Oleh karena itu, dengan menerapkan strategi ini sebanyak $x-y$ kali, Asep akan menang. Terbukti.
        \end{bukti}
    \end{claim*}

    
    Sekarang kita akan membagi kasus berdasarkan $n$ modulo $4$.

\begin{enumerate}
    \item Jika $n=4k\geq 16$, 
    kita dapat langsung menggunakan klaim untuk mendapatkan
    \[f(n) \leq |k-4| = \dfrac{n-16}{4} < \dfrac{n-5}{4}.\]
    
    \item Jika $n=4k+2 \geq 18$,
    pertama-tama kita tambahkan atau kurangkan $2$ untuk mendapatkan $4k$ atau $4k+4$. Kemudian kita terapkan klaim tersebut. Dengan mengikuti strategi ini, kita dapat
    \[f(n) \ leq 1+ |(k+1)-4| = \dfrac{n-10}{4} < \dfrac{n-5}{4}.\]
    
    \item Jika $n=4k+3 \geq 19$ dan langkah pertama Neneng adalah "naik", Asep dapat mengubah bilangan tersebut menjadi $4k+4$. Maka,
    \[f(n) \leq 1+|(k+1)-4| =\dfrac{n-11}{4} < \dfrac{n-5}{4}.\]
    
    \item Jika $n=4k+3 \geq 19$ dan langkah pertama Neneng adalah "turun", Asep dapat mengubah bilangan tersebut menjadi $4k+2$. Maka,
    \[f(n) \leq 1+\dfrac{(n-1)-10}{4} =\dfrac{n-7}{4} < \dfrac{n-5}{4}.\]
    
    \item Jika $n=4k+1 \geq 17$ dan langkah pertama Neneng adalah "turun", 
    Asep dapat mengubah bilangan tersebut menjadi $4k$. Maka,
    \[f(n) \leq 1+|k-4| =\dfrac{n-13}{4} < \dfrac{n-5}{4}.\]
    
    \item Jika $n=4k+1 \geq 17$ dan langkah pertama Neneng adalah "naik", 
    Asep dapat mengubah bilangan tersebut menjadi $4k+2$. Maka,
    \[f(n) \leq 1+\dfrac{(n+1)-10}{4} =\dfrac{n-5}{4}.\]
\end{enumerate}

Sisa kasus lainnya adalah $n=14$, di mana Asep dapat mengganti $14 \to 16$ atau $14 \to 12 \to (9 \text{ atau } 16)$, dan $n=15$, di mana Asep dapat mengganti $15 \to 16$ atau $15 \to 12 \to (9 \text{ atau } 16)$. Terbukti.
\end{solusi}