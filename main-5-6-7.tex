\documentclass[12pt]{scrartcl}
\usepackage[sexy]{evan}
\usepackage{graphicx,amsmath,amssymb,amsthm,amsfonts,babel}
\usepackage{tikz, tkz-euclide}
\usepackage{lipsum}
\usepackage{setspace}
\graphicspath{ {./} }
\usetikzlibrary{calc,through,intersections}
\usepackage[paperwidth=16.8cm, paperheight=16.8cm,margin=0.6cm]{geometry}

\colorlet{EvanRed}{Red!50!Purple}

\newcommand{\siku}[4][.5cm]
	{
	\coordinate (tempa) at ($(#3)!#1!(#2)$);
	\coordinate (tempb) at ($(#3)!#1!(#4)$);
	\coordinate (tempc) at ($(tempa)!0.5!(tempb)$);%midpoint
	\draw[black] (tempa) -- ($(#3)!2!(tempc)$) -- (tempb);
	}
	\usetikzlibrary{calc,positioning,intersections}

\setstretch{1.5}

\usepackage{etoolbox}
\newcommand{\zerodisplayskips}{%
  \setlength{\abovedisplayskip}{5pt}%
  \setlength{\belowdisplayskip}{5pt}%
  \setlength{\abovedisplayshortskip}{5pt}%
  \setlength{\belowdisplayshortskip}{5pt}}
\appto{\normalsize}{\zerodisplayskips}
\appto{\small}{\zerodisplayskips}
\appto{\footnotesize}{\zerodisplayskips}
\setlength\parindent{10pt}

\pagestyle{empty}

\title{Solusi OSN Matematika SMA 2023 part 2}
\author{Nomor 5, 6 dan 7 Hari Kedua}
\date{}

\begin{document}
\maketitle
\newpage
\section{Soal}
\subsection{Nomor 5 Hari Kedua}
Misalkan $a, b$ bilangan asli sehingga $\operatorname{FPB}(a, b)+\operatorname{KPK}(a, b)$ merupakan kelipatan $a+ 1$. Jika $b \le a$, buktikan bahwa $b$ merupakan bilangan kuadrat sempurna. 

\textbf{Catatan:} $\operatorname{FPB}(a, b)$ dan $\operatorname{KPK}(a, b)$ berturut-turut menyatakan faktor persekutuan terbesar dan kelipatan persekutuan terkecil dari $a$ dan $b$.
\subsection{Nomor 6 Hari Kedua}
Tentukan banyaknya permutasi $a_1, a_2, \dots, a_n$ dari $1,2,\dots,n$ sehingga untuk setiap bilangan asli $k$ dengan $1 \le k \le n$ terdapat bilangan bulat $r$ dengan $0 \le r \le n-k$ yang memenuhi
$$1+2+\dots+k = a_{r+1}+a_{r+2}+\dots+a_{r+k}.$$
\subsection{Nomor 7 Hari Kedua}
Diberikan segitiga $ABC$ dengan $\angle ACB = 90^\circ$. Misalkan $\omega$ lingkaran luar $ABC$. Garis singgung terhadap $\omega$ di titik $B$ dan $C$ bertemu di $P$. Misalkan $M$ titik tengah $PB$. Garis $CM$ memotong $\omega$ di $N$ dan garis $PN$ memotong $AB$ di $E$. Titik $D$ pada $CM$ sehingga $ED \parallel BM$. Buktikan bahwa lingkaran luar $CDE$ menyinggung $\omega$.

\section{Solusi}
\subsection{Nomor 5 Hari Kedua}
Misalkan $a, b$ bilangan asli sehingga $\operatorname{FPB}(a, b)+\operatorname{KPK}(a, b)$ merupakan kelipatan $a+ 1$. Jika $b \le a$, buktikan bahwa $b$ merupakan bilangan kuadrat sempurna. 

\textbf{Catatan:} $\operatorname{FPB}(a, b)$ dan $\operatorname{KPK}(a, b)$ berturut-turut menyatakan faktor persekutuan terbesar dan kelipatan persekutuan terkecil dari $a$ dan $b$.
\newline
\textit{(Soal dipropose oleh Reza Wahyu Kumara (Perunggu IMO 2014))}
\begin{motivasi}
    Pendefinisian $a=dx$ dan $b=dy$ dengan $d=\operatorname{FPB}(a,b)$ seringkali menjadi kunci penyelesaian banyak soal, termasuk soal ini. Motivasi munculnya cukup mudah apalagi di soal diberikan syarat keterbagian, sehingga harusnya intuisi secara alami untuk menjadikan $a$ dan $b$ menjadi faktor yang dapat dibagi oleh suatu bilangan $d$.
\end{motivasi}
\begin{solusi}[oleh Azzam L. H.]
    Misalkan $a=dx$ dan $b=dy$ dengan $d,x,y \in \NN$ dan $\operatorname{FPB}(x,y)=1$. Dari sini jelas bahwa $\operatorname{FPB}(a,b)=d$ dan $\operatorname{KPK}(a,b) = dxy$.

    Akan dibuktikan bahwa $d=y$. Andaikan $d \neq y$ kita punya
    \begin{align*}
        a+ 1 &\mid \operatorname{FPB}(a, b)+\operatorname{KPK}(a, b)\\
        dx+1 &\mid d+dxy\\
        dx+1 &\mid d+dxy - y(dx+1)\\
        dx+1 &\mid d-y
    \end{align*}
    yang menyebabkan $dx+1 \le |d-y|$.
    Di sisi lain
    \begin{align*}
        d-y < d < dx + 1 \text{ dan } y-d < y \le dy = b \le a < a+1 = dx+1
    \end{align*}
    yang setara dengan $|d-y| < dx+1$. Tetapi hal ini akan menyebabkan $|d-y| < dx+1 \le |d-y|$, kontradiksi. Oleh karena itu haruslah $d=y$ sehingga mengakibatkan $b=dy=d^2$ merupakan bilangan kuadrat sempurna.
\end{solusi}
\subsection{Nomor 6 Hari Kedua}
Tentukan banyaknya permutasi $a_1, a_2, \dots, a_n$ dari $1,2,\dots,n$ sehingga untuk setiap bilangan asli $k$ dengan $1 \le k \le n$ terdapat bilangan bulat $r$ dengan $0 \le r \le n-k$ yang memenuhi
$$1+2+\dots+k = a_{r+1}+a_{r+2}+\dots+a_{r+k}.$$
\textit{(Soal dipropose oleh Reza Wahyu Kumara (Perunggu IMO 2014))}
\begin{motivasi}
    Setelah saya mencoba-coba beberapa kasus, mulai dari $n=2,3,4,6$ saya mendapatkan suatu pola bahwa banyak solusi berbentuk $2^\text{sesuatu}$. Lalu, dari pola tersebut saya terpikir untuk memakai induksi karena soal persyaratan di soal pasti tergantung oleh kondisi sebelumnya, banyak permutasi pada $n+1$ pasti ada bagian yang merupakan permutasi pada $n$ yang memenuhi. Dari sini tinggal memikirkan bagaimana mengkonstruksi permutasi tersebut.
\end{motivasi}
\begin{solusi}[oleh Azzam L. H.]
    Misalkan tupel $T_n=(a_1,a_2,\dots,a_n)$ yang memenuhi (yang merupakan permutasi dari $(1,2,\dots,n)$) disebut sebagai \textit{kawaii}. Misalkan banyaknya $n$ \textit{kawaii} yang memenuhi adalah $A_n$. Akan dibuktikan dengan induksi di $n$ bahwa $A_n=2^{n-1}$ untuk $n \in \NN$.

    Untuk $n=1$ jelas bahwa $(a_1)=(1)$ sehingga $A_1=1=2^{1-1}$. Untuk $n=2$ jelas bahwa $(a_1,a_2) = (1,2), (2,1)$ sehingga $A_2=2^{2-1}$. 
    
    Andaikan pernyataan benar untuk $n=k \in \NN$, $T_k=(a_1,a_2,\dots,a_k)$ adalah \textit{kawaii} dengan $A_k = 2^{k-1}$. 
    
    Untuk $n=k+1$, misalkan $T_{k+1}=(b_1,b_2,\dots,b_{k+1})$ adalah permutasi yang \textit{kawaii}. Perhatikan karena $T_k$ merupakan permutasi \textit{kawaii}, maka untuk membentuk $T_{k+1}$, kita tidak bisa menyisipkan $k+1$ di antara $a_1,a_2,\dots,a_k$ (tidak bisa menyisipkan di "gap" antara suku-suku sebelumnya). Kita hanya dapat menyisipkan $k+1$ di "kiri" atau "kanan" tupel $T_k$ sehingga $T_{k+1}$ sama dengan $(k+1,a_1,\dots,a_k)$ atau $(a_1,\dots,a_k,k+1)$.  Hal ini karena nilai dari $a_1+\dots+a_k = 1+\dots+k$, jika disisipkan $k+1$ di "gap" antar suku-suku tersebut (selain di kiri dan kanan tupel $T_k$), maka tidak akan ada nilai $r$ yang membuat $b_{r+1}+\dots+b_{r+k}=1+2+\dots+k$.
    
    Oleh karena itu, untuk setiap $T_k$ yang kawaii, akan ada dua buah $T_{k+1}$ yang juga kawaii. Dari sini akan didapat $A_{k+1}=2A_k = 2\cdot 2^{k-1} = 2^{k+1 -1}$. Dari induksi tersebut, terbukti bahwa banyaknya permutasi yang memenuhi soal tersebut adalah $2^{n-1}$.
\end{solusi}
\subsection{Nomor 7 Hari Kedua}
Diberikan segitiga $ABC$ dengan $\angle ACB = 90^\circ$. Misalkan $\omega$ lingkaran luar $ABC$. Garis singgung terhadap $\omega$ di titik $B$ dan $C$ bertemu di $P$. Misalkan $M$ titik tengah $PB$. Garis $CM$ memotong $\omega$ di $N$ dan garis $PN$ memotong $AB$ di $E$. Titik $D$ pada $CM$ sehingga $ED \parallel BM$. Buktikan bahwa lingkaran luar $CDE$ menyinggung $\omega$.
\newline
\textit{(Soal dipropose oleh Reza Wahyu Kumara (Perunggu IMO 2014))}
\begin{motivasi}
    Dari soal yang meminta membuktikan ketersinggungan, saya langsung terpikir dengan \textit{alternate segment theorem}. Dari sini saya berusaha membuktikan bahwa $\angle DEC = \angle MCP$. Nah dari tujuan ini, saya berusaha memaksa untuk \textit{angle chasing} karena sepertinya soal ini bagus untuk permainan sudut, (\textit{feeling}). Lalu, hal ini diperkuat dengan fakta bahwa beberapa sudut bernilai $90^\circ$ karena keadaan diameter lingkaran serta ada sudut yang sama dari dua garis yang saling sejajar. 

    Selain itu, dari keadaan titik $M$ sebagai titik tengah $PB$ dan keadaan $PB$ menyinggung $(ABC)$ di $B$, saya langsung teringat dengan \textit{power of a point}, atau lebih tepatnya sebuah \textit{lemma} di bukunya Evan Chen. Dari situ mudah dicapai tujuan kita untuk mencari kesebangunan dan sudut-sudut yang sama.
\end{motivasi}
\begin{solusi}[oleh Azzam L. H.]
    Notasikan $\dangle$ sebagai \textit{directed angle}. Misalkan garis $DE$ memotong garis $AN$ dan $CP$ berturut-turut di $F$ dan $G$.
    
    Dengan \textit{power of a point} di lingkaran $(ABC)$ kita punya $MP^2=MB^2=MN \cdot MC$ yang menyebabkan $\triangle MNP \sim \triangle MPC$ sehingga $\dangle NPM = \dangle MCP$.
    Selanjutnya, perhatikan bahwa $\dangle BEF = \dangle BNF = 90^\circ$ yang menunjukkan bahwa $FNBE$ siklis. Dengan memadukan hal tersebut dan fakta bahwa $DE \parallel PB$, serta pemakaian \textit{alternate segment theorem} karena $PC$ menyinggung $(ABC)$ di $C$, akan didapat
    \begin{align*}
        \dangle NBF = \dangle NEF = \dangle NPM = \dangle MCP = \dangle NAC = \dangle NBC
    \end{align*}
    yang langsung mengakibatkan $B,F,C$ segaris. 
    \begin{center}
        \begin{asy}
            import olympiad;
            import geometry;
            size(200);
            pair A,B,C;
            A = dir(180);
            B = dir(0);
            C = dir(108);
            draw(B--A--C--B);
            dot("$A$", A, W);
            dot("$B$", B, E);
            dot("$C$", C, NW);  
            circle abc = circumcircle(triangle(A,B,C));
            line tangentB = tangent(abc, B);
            line tangentC = tangent(abc, C);
            pair P = intersectionpoint(tangentB, tangentC);
            dot("$P$", P, N);
            draw(abc, red);
            draw(B--P, blue);
            draw(P--C);
            pair M = (B+P)/2;
            dot("$M$", M, E);
            draw(C--M);
            pair NN[] = intersectionpoints(line(C,M),abc);
            pair N = NN[0];
            dot("$N$", N, N);
            draw(A--N--B);
            pair E = intersectionpoint(line(P,N),line(A,B));
            dot("$E$", E, S);
            draw(P--E);
            draw(C--E);
            pair PAR = E-B+M;
            line ED = line(PAR, E);
            pair D = intersectionpoint(line(C,M),ED);
            dot("$D$", D, N);
            draw(E--D, blue);
            pair G = intersectionpoint(line(C,P),ED);
            dot("$G$", G, NW);
            draw(D--G, blue);
            circle cde = circumcircle(triangle(C,E,D));
            draw(cde, red+dashed);
            pair F = intersectionpoint(line(C,B),line(A,N));
            dot("$F$", F , SW);
            draw(circumcircle(triangle(C,F,E)), blue);
            draw(circumcircle(triangle(N,F,E)), blue);
        \end{asy}
    \end{center}
    Selanjutnya, kita juga punya $CFEA$ siklis karena $\dangle CFE = \dangle CAE = 90^\circ$. Oleh karena itu dengan bantuan \textit{alternate segment theorem} bisa didapat 
    \begin{align*}
        \dangle DEC = \dangle FEC = \dangle FAC = \dangle NAC = \dangle NCP
    \end{align*}
    yang menyebabkan $PC$ menyinggung lingkaran luar $(CDE)$ di $C$. Terbukti.
\end{solusi}

\end{document}
