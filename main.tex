\documentclass[11pt]{scrartcl}
\usepackage{graphicx}
\graphicspath{{./}}
\usepackage[sexy]{evan}
\usepackage[normalem]{ulem}
\usepackage{hyperref}
\usepackage{mathtools}
\hypersetup{
    colorlinks=true,
    linkcolor=blue,
    filecolor=magenta,      
    urlcolor=cyan,
    pdftitle={Overleaf Example},
    pdfpagemode=FullScreen,
    }

\renewcommand{\dangle}{\measuredangle}

\renewcommand{\baselinestretch}{1.5}

\addtolength{\oddsidemargin}{-0.4in}
\addtolength{\evensidemargin}{-0.4in}
\addtolength{\textwidth}{0.8in}
% \addtolength{\topmargin}{-0.2in}
% \addtolength{\textheight}{1in} 

\usepackage[margin=1.5cm]{geometry}


\setlength{\parindent}{0pt}

\usepackage{pgfplots}
\pgfplotsset{compat=1.15}
\usepackage{mathrsfs}
\usetikzlibrary{arrows}

\title{OSN Matematika SMA 2023}
\author{ditulis ulang oleh Azzam Labib}
\date{29-30 Agustus 2023}

\begin{document}

\maketitle
\section{Soal Hari Pertama}
\begin{enumerate}
    \item Diberikan segitiga lancip $ABC$ dengan sisi terpanjang $BC$. Titik $D$,$E$ berturut-turut pada $AC$, $AB$ sehingga $BA=BD$ dan $CA=CE$. Titik $A'$ merupakan pencerminan $A$ terhadap garis $BC$. Buktikan bahwa lingkaran luar segitiga $ABC$ dan $A'DE$ mempunyai panjang jari-jari yang sama.
    \item Tentukan semua fungsi $f: \RR \to \RR$ sehingga untuk setiap bilangan real $x$, $y$ berlaku
    $$f(f(x)+y)=\floor{x+f(f(y))}.$$

\textbf{Catatan:} $\floor{x}$ menyatakan bilangan bulat terbesar yang kurang dari atau sama dengan $x$.
    \item Sebuah bilangan asli $n$ tertulis di papan. Pada setiap langkah, Neneng dan Asep mengubah angka di papan dengan peraturan sebagai berikut: Misalkan angka di papan adalah $X$. Awalnya, Neneng memilih tanda naik atau turun. Kemudian, Asep memilih suatu pembagi positif $d$ dari $X$, dan mengganti $X$ dengan $X+d$ jika Neneng memilih naik, atau $X-d$ jika Neneng memilih turun. Asep menang jika bilangan di papan merupakan suatu bilangan kuadrat sempurna tak nol, dan kalah jika di suatu saat ia menuliskan angka nol. Buktikan jika $n \ge 14$, Asep dapat menang dalam paling banyak $(n-5)/4$ langkah.
    \item Tentukan ada atau tidak bilangan asli $N$ yang memenuhi tiga syarat berikut:
    \begin{itemize}
        \item $N$ habis dibagi $2^{2023}$, tapi tidak habis dibagi $2^{2024}$,
        \item $N$ hanya memuat tiga digit berbeda, dan $N$ tidak memiliki digit nol,
        \item Tepat $99.9\%$ digit $N$ merupakan bilangan ganjil.
    \end{itemize}
\end{enumerate}
\newpage
\section{Soal Hari Kedua}
\begin{enumerate}[resume]
    \item Misalkan $a, b$ bilangan asli sehingga $\operatorname{FPB}(a, b)+\operatorname{KPK}(a, b)$ merupakan kelipatan $a+ 1$. Jika $b \le a$, buktikan bahwa $b$ merupakan bilangan kuadrat sempurna. 

\textbf{Catatan:} $\operatorname{FPB}(a, b)$ dan $\operatorname{KPK}(a, b)$ berturut-turut menyatakan faktor persekutuan terbesar dan kelipatan persekutuan terkecil dari $a$ dan $b$.
    \item Tentukan banyaknya permutasi $a_1, a_2, \dots, a_n$ dari $1,2,\dots,n$ sehingga untuk setiap bilangan asli $k$ dengan $1 \le k \le n$ terdapat bilangan bulat $r$ dengan $0 \le r \le n-k$ yang memenuhi
$$1+2+\dots+k = a_{r+1}+a_{r+2}+\dots+a_{r+k}.$$
    \item Diberikan segitiga $ABC$ dengan $\angle ACB = 90^\circ$. Misalkan $\omega$ lingkaran luar $ABC$. Garis singgung terhadap $\omega$ di titik $B$ dan $C$ bertemu di $P$. Misalkan $M$ titik tengah $PB$. Garis $CM$ memotong $\omega$ di $N$ dan garis $PN$ memotong $AB$ di $E$. Titik $D$ pada $CM$ sehingga $ED \parallel BM$. Buktikan bahwa lingkaran luar $CDE$ menyinggung $\omega$.
    \item Diberikan tiga bilangan bulat positif berbeda $a, b, c$. Definisikan $S(a, b, c)$ sebagai himpunan semua akar rasional dari $px^2 + qx + r = 0$ untuk semua $(p, q, r)$ yang merupakan permutasi dari $(a, b, c)$. Sebagai contoh, $(1, 2, 3) = \{-1,-2, -\frac{1}{2}\}$ karena persamaan $x^2 + 3x + 2 = 0$ memiliki akar $-1$ dan $-2$, persamaan $2x^2 + 3x + 1 = 0$ memiliki akar $-1$ dan $-\frac{1}{2}$, sedangkan untuk permutasi yang lainnya persamaan kuadrat yang terbentuk tidak memiliki akar rasional. Tentukan maksimum banyaknya elemen di $S(a, b, c)$.
\end{enumerate}
\newpage
\section{Solusi Hari Pertama}
\begin{enumerate}
    \item Diberikan segitiga lancip $ABC$ dengan sisi terpanjang $BC$. Titik $D$,$E$ berturut-turut pada $AC$, $AB$ sehingga $BA=BD$ dan $CA=CE$. Titik $A'$ merupakan pencerminan $A$ terhadap garis $BC$. Buktikan bahwa lingkaran luar segitiga $ABC$ dan $A'DE$ mempunyai panjang jari-jari yang sama.
\newline
\textit{(Soal dipropose oleh Reza Wahyu Kumara (Perunggu IMO 2014))}
\begin{motivasi}
    Dari syarat pencerminan $A$ ke $A'$ akan terbentuk segitiga yang kongruen dengan $ABC$ yaitu $A'BC$. Dari sini karena letak $A'DE$ "lebih dekat" dengan $A'BC$ maka saya iseng ingin buktikan kedua lingkaran tersebut ternyata punya satu lingkaran luar yang sama (dan ternyata benar). 
\end{motivasi}
\begin{solusi}[oleh Azzam L. H.]
    Notasikan $\dangle$ sebagai $\textit{directed angle} \mod 180^\circ$.
    Perhatikan karena $A'$ adalah refleksi dari $A$ terhadap $BC$, maka $AC=CA'$ dan $AB=BA'$ sehingga $\triangle ABC \cong \triangle A'BC$. Oleh karena itu, panjang jari-jari lingkaran luar segitiga $ABC$ dan $A'BC$ sama. Dari sini cukup dibuktikan bahwa ternyata lingkaran $(A'BC)$ dan $(A'DE)$ adalah lingkaran yang sama atau cukup dibuktikan bahwa $A',B,E,D,C$ berada di satu lingkaran yang sama.
    \begin{center}
        \begin{asy}
            import geometry;
            import olympiad;
            size(170);
            pair A = (0,1.75);
            pair B = (2,0);
            pair C = (-1,0);
            real ab = length(A-B);
            pair AD[] = intersectionpoints(line(A,C),circle(B,ab));
            pair D = AD[0];
            real ac = length(A-C);
            pair AE[] = intersectionpoints(line(A,B),circle(C,ac));
            pair E = AE[1];
            pair A1 = reflect(B,C)*A;
            circle abc = circumcircle(triangle(A,B,C));
            circle dea1 = circumcircle(triangle(A1,D,E));
            dot("$A$", A, N);
            dot("$B$", B, E);
            dot("$C$", C, W);
            dot("$D$", D, NW);
            dot("$E$", E, NE);
            dot("$A'$", A1, SW);
            draw(B--C--A1--B);
            draw(A--B--D, blue);
            draw(A--C--E, red);
            draw(abc);
            draw(dea1, dashed);
        \end{asy}
    \end{center}
    Perhatikan bahwa karena $AB=BD$ dan $AC=CE$ maka 
    \begin{align*}
        \dangle BDC = \dangle BDA = \dangle DAE = \dangle AEC = \dangle BEC
    \end{align*}
    yang menunjukkan $B,D,E,C$ berada di satu lingkaran. Di sisi lain 
    \begin{align*}
        \dangle BEC = \dangle AEC = \dangle CAB = \dangle BA'C
    \end{align*}
    yang menunjukkan $A'B,E,C$ berada di satu lingkaran. Dari kedua fakta tersebut, dapat disimpulkan bahwa $A',B,E,D,C$ berada di satu lingkaran yang sama, sesuai yang ingin dibuktikan.
\end{solusi}
    \newpage
    \item Tentukan semua fungsi $f: \RR \to \RR$ sehingga untuk setiap bilangan real $x$, $y$ berlaku
    $$f(f(x)+y)=\floor{x+f(f(y))}.$$

\textbf{Catatan:} $\floor{x}$ menyatakan bilangan bulat terbesar yang kurang dari atau sama dengan $x$.
\newline
\textit{(Soal di\textit{propose} oleh Valentio Iverson (Absolute Winner, Emas OSN 2019))}
\begin{motivasi}
    Jika dilihat pada soal, ekspresi $f(\dots) = \floor{\dots}$ menandakan bahwa $f$ bernilai bulat untuk semua $y$, oleh karena itu aman dikatakan bahwa ini adalah fungsi yang bernilai bulat. Lalu, saya sendiri coba "memaksa" agar $f(0)$ atau ekspresi yang mirip ditemukan agar mungkin bisa mendapatkan ekspresi yang lebih bagus lagi. Selain itu, dari ekspresi di soal saya mempunyai tujuan untuk membuktikan bahwa $f(\dots) = \floor{\dots} + \text{sesuatu}$.
\end{motivasi}
\begin{solusi}[oleh Azzam L. H.]
    Notasikan $P(x,y)$ sebagai asersi $x$ dan $y$ pada persamaan di soal. Pertama jika diasersi $P(x,-f(x)+y)$ untuk sebarang $x,y \in \RR$ maka 
    \begin{align*}
        f(y) = \floor{x + f(f(-f(x)+y))}
    \end{align*}
    yang menunjukkan bahwa $f: \RR \to \ZZ$ atau hasil pemetaan $f$ harus bulat.
    Selanjutnya, karena $f(f(0))$ bulat, dengan asersi $P(x,0)$ untuk sembarang $x \in \RR$ akan didapat
    \begin{align*}
        f(f(x)) = \floor{x+f(f(0))} = \floor{x}+f(f(0))
    \end{align*}
    oleh karena itu, jika diasersi $P(x,-f(x))$ untuk sembarang $x \in \RR$ dengan persamaan sebelumnya (ganti $x$ dengan $-f(x)$) dan karena hasil $f$ bulat, maka akan didapat
    \begin{align*}
        f(0) &= \floor{x+f(f(-f(x)))}\\
        f(0) &= \floor{x} + f(f(-f(x)))\\
        f(0) &= \floor{x} + \floor{-f(x)} + f(f(0))\\
        f(0) &= \floor{x} -f(x) + f(f(0))\\
        f(x) &= \floor{x} + f(f(0))-f(0)
    \end{align*}
    Karena $f(0)$ konstan dan bulat, maka $t = f(f(0))-f(0)$ juga bernilai konstan dan bulat sehingga $f(x) = \floor{x}+t$. Jika dicek ke soal,
    \begin{align*}
        f(f(x)+y)&=f(\floor{x}+t+y)\\
                &=\floor{\floor{x}+t+y}+t\\
                &=\floor{\floor{x}}+\floor{y}+t+t\\
                &= \floor{\floor{x}} + \floor{\floor{y}+t+t}\\
                &= \floor{\floor{x}+\floor{\floor{y}+t+t}}\\
                &= \floor{x+f(\floor{y}+t)}\\
                &= \floor{x+f(f(y))}
    \end{align*}
    ternyata $\boxed{f(x) = \floor{x} + t}$ untuk sembarang $x \in \RR$ dan konstanta $t \in \ZZ$ adalah fungsi yang memenuhi soal. 
    
\end{solusi}
    \newpage
    \item Sebuah bilangan asli $n$ tertulis di papan. Pada setiap langkah, Neneng dan Asep mengubah angka di papan dengan peraturan sebagai berikut: Misalkan angka di papan adalah $X$. Awalnya, Neneng memilih tanda naik atau turun. Kemudian, Asep memilih suatu pembagi positif $d$ dari $X$, dan mengganti $X$ dengan $X+d$ jika Neneng memilih naik, atau $X-d$ jika Neneng memilih turun. Asep menang jika bilangan di papan merupakan suatu bilangan kuadrat sempurna tak nol, dan kalah jika di suatu saat ia menuliskan angka nol. Buktikan jika $n \ge 14$, Asep dapat menang dalam paling banyak $(n-5)/4$ langkah.
\newline
\textit{(Soal di\textit{propose} oleh Muhammad Afifurrahman (Emas OSN 2015))}
\begin{motivasi}(oleh Muhammad Afifurrahman)
    \begin{itemize}
        \item Motivasi untuk solusi ini berasal dari $f(n(n+1))=1$ (karena Asep dapat berpindah ke $n^2$ atau $(n+1)^2$ dalam kasus ini).
\item \textit{case work}nya tidak terlalu berat jika sudah memiliki klaim tersebut.
\item Dengan menguji semua kemungkinan, seseorang dapat membuktikan bahwa $f(17)=3$ yang menghasilkan sebuah kasus ketaksamaan.
\item Tentu saja, batasan klaim tersebut tidak optimal. Namun, klaim tersebut sudah cukup untuk membuktikan pernyataan di soal.
\item Selain itu, batasan masalah ini tidak optimal untuk nilai $n$ yang besar; khususnya, dengan memodifikasi (dan mempertimbangkan lebih banyak kasus modulo...), ada $C(k)$ sedemikian sehingga $f(n)\leq (n-C(k))/k$ untuk semua nilai $n$ yang cukup besar.
\item Lebih lanjut, teman saya menunjukkan bahwa $f(n) \leq \lceil \log_2 n \rceil$ untuk semua nilai $n$ yang cukup besar, dengan memperhatikan bahwa $f(2^{k})=1$ ketika $k$ ganjil. 
    \end{itemize}
\end{motivasi}
\begin{solusi}(oleh Muhammad Afifurrahman)
    Misalkan $f(n)$ menotasikan langkah terkecil yang dibutuhkan Asep untuk menang. 

    \begin{claim*}
        $f(xy)\leq |x-y|$. 
        \begin{bukti}
            \item Jika $x=y$, Asep sudah selesai. Andaikan $x \neq y$ , WLOG $x>y$, Asep dapat melanjutkan seperti berikut ini:
            \item Jika Neneng memilih naik, Asep dapat mengganti $xy$ dengan $x(y+1)$.
            \item Jika Neneng memilih turun, Asep dapat mengganti $xy$ dengan $(x-1)y$. Langkah ini valid karena $x\geq 2$.
            Dalam setiap langkah, Asep akan mengurangi selisih antara kedua bilangan tersebut sebanyak tepat 1. Oleh karena itu, dengan menerapkan strategi ini sebanyak $x-y$ kali, Asep akan menang. Terbukti.
        \end{bukti}
    \end{claim*}

    
    Sekarang kita akan membagi kasus berdasarkan $n$ modulo $4$.

\begin{enumerate}
    \item Jika $n=4k\geq 16$, 
    kita dapat langsung menggunakan klaim untuk mendapatkan
    \[f(n) \leq |k-4| = \dfrac{n-16}{4} < \dfrac{n-5}{4}.\]
    
    \item Jika $n=4k+2 \geq 18$,
    pertama-tama kita tambahkan atau kurangkan $2$ untuk mendapatkan $4k$ atau $4k+4$. Kemudian kita terapkan klaim tersebut. Dengan mengikuti strategi ini, kita dapat
    \[f(n) \ leq 1+ |(k+1)-4| = \dfrac{n-10}{4} < \dfrac{n-5}{4}.\]
    
    \item Jika $n=4k+3 \geq 19$ dan langkah pertama Neneng adalah "naik", Asep dapat mengubah bilangan tersebut menjadi $4k+4$. Maka,
    \[f(n) \leq 1+|(k+1)-4| =\dfrac{n-11}{4} < \dfrac{n-5}{4}.\]
    
    \item Jika $n=4k+3 \geq 19$ dan langkah pertama Neneng adalah "turun", Asep dapat mengubah bilangan tersebut menjadi $4k+2$. Maka,
    \[f(n) \leq 1+\dfrac{(n-1)-10}{4} =\dfrac{n-7}{4} < \dfrac{n-5}{4}.\]
    
    \item Jika $n=4k+1 \geq 17$ dan langkah pertama Neneng adalah "turun", 
    Asep dapat mengubah bilangan tersebut menjadi $4k$. Maka,
    \[f(n) \leq 1+|k-4| =\dfrac{n-13}{4} < \dfrac{n-5}{4}.\]
    
    \item Jika $n=4k+1 \geq 17$ dan langkah pertama Neneng adalah "naik", 
    Asep dapat mengubah bilangan tersebut menjadi $4k+2$. Maka,
    \[f(n) \leq 1+\dfrac{(n+1)-10}{4} =\dfrac{n-5}{4}.\]
\end{enumerate}

Sisa kasus lainnya adalah $n=14$, di mana Asep dapat mengganti $14 \to 16$ atau $14 \to 12 \to (9 \text{ atau } 16)$, dan $n=15$, di mana Asep dapat mengganti $15 \to 16$ atau $15 \to 12 \to (9 \text{ atau } 16)$. Terbukti.
\end{solusi}
    \newpage
    \item Tentukan ada atau tidak bilangan asli $N$ yang memenuhi tiga syarat berikut:
    \begin{itemize}
        \item $N$ habis dibagi $2^{2023}$, tapi tidak habis dibagi $2^{2024}$,
        \item $N$ hanya memuat tiga digit berbeda, dan $N$ tidak memiliki digit nol,
        \item Tepat $99.9\%$ digit $N$ merupakan bilangan ganjil.
    \end{itemize}
\textit{(Soal di\textit{propose} oleh Muhammad Afifurrahman (Emas OSN 2015))}
\begin{motivasi}(oleh Muhammad Afifurrahman)
    Inti utama dari masalah ini (ketika saya memikirkan soal ini) adalah klaim dalam solusi tersebut. Bagian yang paling sulit setelah membuktikan klaim-klaim tersebut adalah untuk menyembunyikan klaim tersebut dalam soal dan pastinya $99.9 \%$ itu cuma jadi pengalih perhatian; setiap pecahan pada interval $(0,1)$ harusnya bisa (dan mungkin juga untuk nol?). Selain itu, dengan menambahkan beberapa digit di depan, ada tak terhingga banyaknya $N$ yang memenuhi pernyataan tersebut.
\end{motivasi}

\begin{solusi}(oleh Muhammad Afifurrahman)
    Pertama akan dibuktikan klaim yang lebih kuat.
    
    \begin{claim*}
        Terdapat barisan bilangan bulat positif $(a_i)_{i=3}^\infty$ sehingga pernyataan berikut berlaku: 
        \begin{itemize}
            \item Untuk semua $n\geq 3$, $a_n$ hanya mengandung digit $1, 2, 3,$
            
            \item $a_n$ memiliki tepat $n$ digit,
            
            \item $v_2(a_n)=n$, dan 
            
            \item semua digit pada $a_n$, kecuali digit terakhir, bernilai ganjil.
        \end{itemize}
        \begin{bukti}
            Akan dilakukan induksi pada $n$. 
            
            Untuk $n=3$, ambil $a_3=312$. Misalkan ada $a_k$ yang memenuhi untuk suatu $k\geq 3$. Ternyata, baik $10^k+a_k$ maupun $3 \cdot 10^k+a_k$ dapat dibagi oleh $2^{k+1}$, dikarenakan
            
            \[ 10^k+a_k \equiv 2^k + 2^k \equiv 0 \pmod {2^{k+1}}, \]
            
            dan salah satunya tidak dapat dibagi oleh $2^{k+2}$ (karena selisih kedua bilangan tersebut adalah $2 \cdot 10^k$ yang tidak dapat dibagi oleh $2^{k+2}$). Oleh karena itu, dapat dipilih $a_{k+1}$ sebagai salah satu dari dua bilangan tersebut. Klaim terbukti,
        \end{bukti}
    \end{claim*}

    Dari klaim tersebut, maka kita dapat mengkonstruksi $N$ dengan $3000$ digit yang memenuhi syarat di soal. Ambil $2023$ digit terakhir dari $N$ sebagai $a_{2023}$ yang telah dikonstruksi sebelumnya, dan ambil 977 digit pertama dari $N$ sebagai $3 \ldots 322$, dengan 975 digit 3 dan dua digit 2. Dengan konstruksi $a_n$, dapat dilihat bahwa $N$ memiliki tiga digit genap yang semuanya adalah 2, dan 2997 digit ganjil yang merupakan 1 atau 3. Oleh karena itu, konstruksi ini memenuhi persyaratan persentase dan digit. 
    Selanjutnya, kita punya
    \[ N = 3\ldots 322 \cdot 10^{2023} + a_{2023}. \]
    Suku pertama dapat dibagi oleh $2^{2024}$, dan suku kedua dapat dibagi oleh $2^{2023}$, tetapi tidak dapat dibagi oleh $2^{2024}$. Oleh karena itu, $v_2(N)=2023$. Konstruksi tersebut memenuhi. Dapat disimpulkan bahwa bilangan $N$ yang dimaksud ada. Terbukti
\end{solusi}
    \newpage
\end{enumerate}

\section{Solusi Hari Kedua}
\begin{enumerate}[resume]
    \item Misalkan $a, b$ bilangan asli sehingga $\operatorname{FPB}(a, b)+\operatorname{KPK}(a, b)$ merupakan kelipatan $a+ 1$. Jika $b \le a$, buktikan bahwa $b$ merupakan bilangan kuadrat sempurna. 

\textbf{Catatan:} $\operatorname{FPB}(a, b)$ dan $\operatorname{KPK}(a, b)$ berturut-turut menyatakan faktor persekutuan terbesar dan kelipatan persekutuan terkecil dari $a$ dan $b$.
\newline
\textit{(Soal dipropose oleh Reza Wahyu Kumara (Perunggu IMO 2014))}
\begin{motivasi}
    Pendefinisian $a=dx$ dan $b=dy$ dengan $d=\operatorname{FPB}(a,b)$ seringkali menjadi kunci penyelesaian banyak soal, termasuk soal ini. Motivasi munculnya cukup mudah apalagi di soal diberikan syarat keterbagian, sehingga harusnya intuisi secara alami untuk menjadikan $a$ dan $b$ menjadi faktor yang dapat dibagi oleh suatu bilangan $d$.
\end{motivasi}
\begin{solusi}[oleh Azzam L. H.]
    Misalkan $a=dx$ dan $b=dy$ dengan $d,x,y \in \NN$ dan $\operatorname{FPB}(x,y)=1$. Dari sini jelas bahwa $\operatorname{FPB}(a,b)=d$ dan $\operatorname{KPK}(a,b) = dxy$.

    Akan dibuktikan bahwa $d=y$. Andaikan $d \neq y$ kita punya
    \begin{align*}
        a+ 1 &\mid \operatorname{FPB}(a, b)+\operatorname{KPK}(a, b)\\
        dx+1 &\mid d+dxy\\
        dx+1 &\mid d+dxy - y(dx+1)\\
        dx+1 &\mid d-y
    \end{align*}
    yang menyebabkan $dx+1 \le |d-y|$.
    Di sisi lain
    \begin{align*}
        d-y < d < dx + 1 \text{ dan } y-d < y \le dy = b \le a < a+1 = dx+1
    \end{align*}
    yang setara dengan $|d-y| < dx+1$. Tetapi hal ini akan menyebabkan $|d-y| < dx+1 \le |d-y|$, kontradiksi. Oleh karena itu haruslah $d=y$ sehingga mengakibatkan $b=dy=d^2$ merupakan bilangan kuadrat sempurna.
\end{solusi}
    \newpage
    \item Tentukan banyaknya permutasi $a_1, a_2, \dots, a_n$ dari $1,2,\dots,n$ sehingga untuk setiap bilangan asli $k$ dengan $1 \le k \le n$ terdapat bilangan bulat $r$ dengan $0 \le r \le n-k$ yang memenuhi
$$1+2+\dots+k = a_{r+1}+a_{r+2}+\dots+a_{r+k}.$$
\textit{(Soal dipropose oleh Reza Wahyu Kumara (Perunggu IMO 2014))}
\begin{motivasi}
    Setelah saya mencoba-coba beberapa kasus, mulai dari $n=2,3,4,6$ saya mendapatkan suatu pola bahwa banyak solusi berbentuk $2^\text{sesuatu}$. Lalu, dari pola tersebut saya terpikir untuk memakai induksi karena soal persyaratan di soal pasti tergantung oleh kondisi sebelumnya, banyak permutasi pada $n+1$ pasti ada bagian yang merupakan permutasi pada $n$ yang memenuhi. Dari sini tinggal memikirkan bagaimana mengkonstruksi permutasi tersebut.
\end{motivasi}
\begin{solusi}[oleh Azzam L. H.]
    Misalkan tupel $T_n=(a_1,a_2,\dots,a_n)$ yang memenuhi (yang merupakan permutasi dari $(1,2,\dots,n)$) disebut sebagai \textit{kawaii}. Misalkan banyaknya $n$ \textit{kawaii} yang memenuhi adalah $A_n$. Akan dibuktikan dengan induksi di $n$ bahwa $A_n=2^{n-1}$ untuk $n \in \NN$.

    Untuk $n=1$ jelas bahwa $(a_1)=(1)$ sehingga $A_1=1=2^{1-1}$. Untuk $n=2$ jelas bahwa $(a_1,a_2) = (1,2), (2,1)$ sehingga $A_2=2^{2-1}$. 
    
    Andaikan pernyataan benar untuk $n=k \in \NN$, $T_k=(a_1,a_2,\dots,a_k)$ adalah \textit{kawaii} dengan $A_k = 2^{k-1}$. 
    
    Untuk $n=k+1$, misalkan $T_{k+1}=(b_1,b_2,\dots,b_{k+1})$ adalah permutasi yang \textit{kawaii}. Perhatikan karena $T_k$ merupakan permutasi \textit{kawaii}, maka untuk membentuk $T_{k+1}$, kita tidak bisa menyisipkan $k+1$ di antara $a_1,a_2,\dots,a_k$ (tidak bisa menyisipkan di "gap" antara suku-suku sebelumnya). Kita hanya dapat menyisipkan $k+1$ di "kiri" atau "kanan" tupel $T_k$ sehingga $T_{k+1}$ sama dengan $(k+1,a_1,\dots,a_k)$ atau $(a_1,\dots,a_k,k+1)$.  Hal ini karena nilai dari $a_1+\dots+a_k = 1+\dots+k$, jika disisipkan $k+1$ di "gap" antar suku-suku tersebut (selain di kiri dan kanan tupel $T_k$), maka tidak akan ada nilai $r$ yang membuat $b_{r+1}+\dots+b_{r+k}=1+2+\dots+k$.
    
    Oleh karena itu, untuk setiap $T_k$ yang kawaii, akan ada dua buah $T_{k+1}$ yang juga kawaii. Dari sini akan didapat $A_{k+1}=2A_k = 2\cdot 2^{k-1} = 2^{k+1 -1}$. Dari induksi tersebut, terbukti bahwa banyaknya permutasi yang memenuhi soal tersebut adalah $2^{n-1}$.
\end{solusi}
    \newpage
    \item Diberikan segitiga $ABC$ dengan $\angle ACB = 90^\circ$. Misalkan $\omega$ lingkaran luar $ABC$. Garis singgung terhadap $\omega$ di titik $B$ dan $C$ bertemu di $P$. Misalkan $M$ titik tengah $PB$. Garis $CM$ memotong $\omega$ di $N$ dan garis $PN$ memotong $AB$ di $E$. Titik $D$ pada $CM$ sehingga $ED \parallel BM$. Buktikan bahwa lingkaran luar $CDE$ menyinggung $\omega$.
\newline
\textit{(Soal dipropose oleh Reza Wahyu Kumara (Perunggu IMO 2014))}
\begin{motivasi}
    Dari soal yang meminta membuktikan ketersinggungan, saya langsung terpikir dengan \textit{alternate segment theorem}. Dari sini saya berusaha membuktikan bahwa $\angle DEC = \angle MCP$. Nah dari tujuan ini, saya berusaha memaksa untuk \textit{angle chasing} karena sepertinya soal ini bagus untuk permainan sudut, (\textit{feeling}). Lalu, hal ini diperkuat dengan fakta bahwa beberapa sudut bernilai $90^\circ$ karena keadaan diameter lingkaran serta ada sudut yang sama dari dua garis yang saling sejajar. 

    Selain itu, dari keadaan titik $M$ sebagai titik tengah $PB$ dan keadaan $PB$ menyinggung $(ABC)$ di $B$, saya langsung teringat dengan \textit{power of a point}, atau lebih tepatnya sebuah \textit{lemma} di bukunya Evan Chen. Dari situ mudah dicapai tujuan kita untuk mencari kesebangunan dan sudut-sudut yang sama.
\end{motivasi}
\begin{solusi}[oleh Azzam L. H.]
    Notasikan $\dangle$ sebagai \textit{directed angle}. Misalkan garis $DE$ memotong garis $AN$ dan $CP$ berturut-turut di $F$ dan $G$.
    
    Dengan \textit{power of a point} di lingkaran $(ABC)$ kita punya $MP^2=MB^2=MN \cdot MC$ yang menyebabkan $\triangle MNP \sim \triangle MPC$ sehingga $\dangle NPM = \dangle MCP$.
    Selanjutnya, perhatikan bahwa $\dangle BEF = \dangle BNF = 90^\circ$ yang menunjukkan bahwa $FNBE$ siklis. Dengan memadukan hal tersebut dan fakta bahwa $DE \parallel PB$, serta pemakaian \textit{alternate segment theorem} karena $PC$ menyinggung $(ABC)$ di $C$, akan didapat
    \begin{align*}
        \dangle NBF = \dangle NEF = \dangle NPM = \dangle MCP = \dangle NAC = \dangle NBC
    \end{align*}
    yang langsung mengakibatkan $B,F,C$ segaris. 
    \begin{center}
        \begin{asy}
            import olympiad;
            import geometry;
            size(200);
            pair A,B,C;
            A = dir(180);
            B = dir(0);
            C = dir(108);
            draw(B--A--C--B);
            dot("$A$", A, W);
            dot("$B$", B, E);
            dot("$C$", C, NW);  
            circle abc = circumcircle(triangle(A,B,C));
            line tangentB = tangent(abc, B);
            line tangentC = tangent(abc, C);
            pair P = intersectionpoint(tangentB, tangentC);
            dot("$P$", P, N);
            draw(abc, red);
            draw(B--P, blue);
            draw(P--C);
            pair M = (B+P)/2;
            dot("$M$", M, E);
            draw(C--M);
            pair NN[] = intersectionpoints(line(C,M),abc);
            pair N = NN[0];
            dot("$N$", N, N);
            draw(A--N--B);
            pair E = intersectionpoint(line(P,N),line(A,B));
            dot("$E$", E, S);
            draw(P--E);
            draw(C--E);
            pair PAR = E-B+M;
            line ED = line(PAR, E);
            pair D = intersectionpoint(line(C,M),ED);
            dot("$D$", D, N);
            draw(E--D, blue);
            pair G = intersectionpoint(line(C,P),ED);
            dot("$G$", G, NW);
            draw(D--G, blue);
            circle cde = circumcircle(triangle(C,E,D));
            draw(cde, red+dashed);
            pair F = intersectionpoint(line(C,B),line(A,N));
            dot("$F$", F , SW);
            draw(circumcircle(triangle(C,F,E)), blue);
            draw(circumcircle(triangle(N,F,E)), blue);
        \end{asy}
    \end{center}
    Selanjutnya, kita juga punya $CFEA$ siklis karena $\dangle CFE = \dangle CAE = 90^\circ$. Oleh karena itu dengan bantuan \textit{alternate segment theorem} bisa didapat 
    \begin{align*}
        \dangle DEC = \dangle FEC = \dangle FAC = \dangle NAC = \dangle NCP
    \end{align*}
    yang menyebabkan $PC$ menyinggung lingkaran luar $(CDE)$ di $C$. Terbukti.
\end{solusi}
    \newpage
    \item Diberikan tiga bilangan bulat positif berbeda $a, b, c$. Definisikan $S(a, b, c)$ sebagai himpunan semua akar rasional dari $px^2 + qx + r = 0$ untuk semua $(p, q, r)$ yang merupakan permutasi dari $(a, b, c)$. Sebagai contoh, $(1, 2, 3) = \{-1,-2, -\frac{1}{2}\}$ karena persamaan $x^2 + 3x + 2 = 0$ memiliki akar $-1$ dan $-2$, persamaan $2x^2 + 3x + 1 = 0$ memiliki akar $-1$ dan $-\frac{1}{2}$, sedangkan untuk permutasi yang lainnya persamaan kuadrat yang terbentuk tidak memiliki akar rasional. Tentukan maksimum banyaknya elemen di $S(a, b, c)$.
\newline
\textit{(Soal di\textit{propose} oleh Valentio Iverson (Absolute Winner, Emas OSN 2019))}
\begin{motivasi}
    (oleh Valentio Iverson) Kesulitan utama dari soal ini terletak pada konstruksi, yang menurut saya sebenarnya cukup masuk akal. 
    
    Kita bisa mulai dengan mengerjakan kasus-kasus untuk $\min(a,b,c)$, sehingga dapat dibuktikan bahwa $\min(a,b,c) = 1$ tidak berhasil.
    
    Kemudian, coba $\min(a,b,c) = 2$, atau ketika kita menginginkan agar $b^2 - 8c$ dan $c^2 - 8b$ adalah kuadrat sempurna. Intuisi yang paling "natural" untuk dicoba adalah memaksa $b$ menjadi polinomial dalam $c$ sehingga $b^2 - 8c$ adalah kuadrat dari polinomial dalam $c$, yang berarti kita ingin sesuatu yang berbentuk seperti $b = 2c + 1$. Kemudian sisanya gampang, coba paksa agar $c^2 - 8(2c + 1) = (c - 8)^2 - 72$ menjadi kuadrat sempurna.
\end{motivasi}
\begin{solusi}(oleh Valentio Iverson)
    Jawabannya adalah $\boxed{8}$, yang tercapai saat $(a,b,c) = (2,17,35)$. Dapat dilihat bahwa pada konstruksi tersebut,
    \[ S(a,b,c) = \left \{-17, - 5,  -\frac{7}{2},  -2, - \frac{1}{2},  - \frac{1}{5}, - \frac{2}{7}, - \frac{1}{17}  \right \}. \]
    
    Untuk membuktikan bahwa batasan tersebut adalah yang paling besar, dapat dilihat agar $px^2 + qx + r$ memiliki solusi rasional, maka haruslah $q^2 - 4pr \ge 0$.
    
    Perhatikan bahwa ada permutasi $(d,e,f)$ dari $(a,b,c)$ sehingga $d < e < f$. 
    
    Dalam kasus ini, $d^2 - 4ef < 0$ dan oleh karena itu baik $ex^2 + dx + f$ maupun $fx^2 + dx + e$ tidak memberikan akar yang rasional. 
    
    Oleh karena itu, paling banyak hanya ada $2 \cdot 4 = 8$ akar rasional dari $4$ persamaan kuadrat yang tersisa, seperti yang ingin dibuktikan.
\end{solusi}
\end{enumerate}


\end{document}
