\documentclass[12pt]{scrartcl}
\usepackage[sexy]{evan}
\usepackage{graphicx,amsmath,amssymb,amsthm,amsfonts,babel}
\usepackage{tikz, tkz-euclide}
\usepackage{lipsum}
\usepackage{setspace}
\graphicspath{ {./} }
\usetikzlibrary{calc,through,intersections}
\usepackage[paperwidth=17cm, paperheight=17cm,margin=0.4cm]{geometry}

\colorlet{EvanRed}{Red!50!Purple}

\newcommand{\siku}[4][.5cm]
	{
	\coordinate (tempa) at ($(#3)!#1!(#2)$);
	\coordinate (tempb) at ($(#3)!#1!(#4)$);
	\coordinate (tempc) at ($(tempa)!0.5!(tempb)$);%midpoint
	\draw[black] (tempa) -- ($(#3)!2!(tempc)$) -- (tempb);
	}
	\usetikzlibrary{calc,positioning,intersections}

\setstretch{1.5}

\usepackage{etoolbox}
\newcommand{\zerodisplayskips}{%
  \setlength{\abovedisplayskip}{5pt}%
  \setlength{\belowdisplayskip}{5pt}%
  \setlength{\abovedisplayshortskip}{5pt}%
  \setlength{\belowdisplayshortskip}{5pt}}
\appto{\normalsize}{\zerodisplayskips}
\appto{\small}{\zerodisplayskips}
\appto{\footnotesize}{\zerodisplayskips}
\setlength\parindent{10pt}

\pagestyle{empty}

\title{Solusi OSN Matematika SMA 2023\\ part 3}
\author{Nomor 3, 4 (Hari Pertama) dan 8 (Hari Kedua)}
\date{}

\begin{document}
\maketitle
\newpage
\section{Soal}
\subsection{Nomor 3 Hari Pertama}
Sebuah bilangan asli $n$ tertulis di papan. Pada setiap langkah, Neneng dan Asep mengubah angka di papan dengan peraturan sebagai berikut: Misalkan angka di papan adalah $X$. Awalnya, Neneng memilih tanda naik atau turun. Kemudian, Asep memilih suatu pembagi positif $d$ dari $X$, dan mengganti $X$ dengan $X+d$ jika Neneng memilih naik, atau $X-d$ jika Neneng memilih turun. Asep menang jika bilangan di papan merupakan suatu bilangan kuadrat sempurna tak nol, dan kalah jika di suatu saat ia menuliskan angka nol. Buktikan jika $n \ge 14$, Asep dapat menang dalam paling banyak $(n-5)/4$ langkah.
\subsection{Nomor 4 Hari Pertama}
Tentukan ada atau tidak bilangan asli $N$ yang memenuhi tiga syarat berikut:
    \begin{itemize}
        \item $N$ habis dibagi $2^{2023}$, tapi tidak habis dibagi $2^{2024}$,
        \item $N$ hanya memuat tiga digit berbeda, dan $N$ tidak memiliki digit nol,
        \item Tepat $99.9\%$ digit $N$ merupakan bilangan ganjil.
    \end{itemize}
\subsection{Nomor 8 Hari Kedua}
Diberikan tiga bilangan bulat positif berbeda $a, b, c$. Definisikan $S(a, b, c)$ sebagai himpunan semua akar rasional dari $px^2 + qx + r = 0$ untuk semua $(p, q, r)$ yang merupakan permutasi dari $(a, b, c)$. Sebagai contoh, $(1, 2, 3) = \{-1,-2, -\frac{1}{2}\}$ karena persamaan $x^2 + 3x + 2 = 0$ memiliki akar $-1$ dan $-2$, persamaan $2x^2 + 3x + 1 = 0$ memiliki akar $-1$ dan $-\frac{1}{2}$, sedangkan untuk permutasi yang lainnya persamaan kuadrat yang terbentuk tidak memiliki akar rasional. Tentukan maksimum banyaknya elemen di $S(a, b, c)$.

\newpage

\section{Solusi}
\subsection{Nomor 3 Hari Pertama}
Sebuah bilangan asli $n$ tertulis di papan. Pada setiap langkah, Neneng dan Asep mengubah angka di papan dengan peraturan sebagai berikut: Misalkan angka di papan adalah $X$. Awalnya, Neneng memilih tanda naik atau turun. Kemudian, Asep memilih suatu pembagi positif $d$ dari $X$, dan mengganti $X$ dengan $X+d$ jika Neneng memilih naik, atau $X-d$ jika Neneng memilih turun. Asep menang jika bilangan di papan merupakan suatu bilangan kuadrat sempurna tak nol, dan kalah jika di suatu saat ia menuliskan angka nol. Buktikan jika $n \ge 14$, Asep dapat menang dalam paling banyak $(n-5)/4$ langkah.
\newline
\textit{(Soal di\textit{propose} oleh Muhammad Afifurrahman (Emas OSN 2015))}
\begin{motivasi}(oleh Muhammad Afifurrahman)
    \begin{itemize}
        \item Motivasi untuk solusi ini berasal dari $f(n(n+1))=1$ (karena Asep dapat berpindah ke $n^2$ atau $(n+1)^2$ dalam kasus ini).
\item \textit{case work}nya tidak terlalu berat jika sudah memiliki klaim tersebut.
\item Dengan menguji semua kemungkinan, seseorang dapat membuktikan bahwa $f(17)=3$ yang menghasilkan sebuah kasus ketaksamaan.
\item Tentu saja, batasan klaim tersebut tidak optimal. Namun, klaim tersebut sudah cukup untuk membuktikan pernyataan di soal.
\item Selain itu, batasan masalah ini tidak optimal untuk nilai $n$ yang besar; khususnya, dengan memodifikasi (dan mempertimbangkan lebih banyak kasus modulo...), ada $C(k)$ sedemikian sehingga $f(n)\leq (n-C(k))/k$ untuk semua nilai $n$ yang cukup besar.
\item Lebih lanjut, teman saya menunjukkan bahwa $f(n) \leq \lceil \log_2 n \rceil$ untuk semua nilai $n$ yang cukup besar, dengan memperhatikan bahwa $f(2^{k})=1$ ketika $k$ ganjil. 
    \end{itemize}
\end{motivasi}
\begin{solusi}(oleh Muhammad Afifurrahman)
    Misalkan $f(n)$ menotasikan langkah terkecil yang dibutuhkan Asep untuk menang. 

    \begin{claim*}
        $f(xy)\leq |x-y|$. 
        \begin{bukti}
            \item Jika $x=y$, Asep sudah selesai. Andaikan $x \neq y$ , WLOG $x>y$, Asep dapat melanjutkan seperti berikut ini:
            \item Jika Neneng memilih naik, Asep dapat mengganti $xy$ dengan $x(y+1)$.
            \item Jika Neneng memilih turun, Asep dapat mengganti $xy$ dengan $(x-1)y$. Langkah ini valid karena $x\geq 2$.
            Dalam setiap langkah, Asep akan mengurangi selisih antara kedua bilangan tersebut sebanyak tepat 1. Oleh karena itu, dengan menerapkan strategi ini sebanyak $x-y$ kali, Asep akan menang. Terbukti.
        \end{bukti}
    \end{claim*}

    
    Sekarang kita akan membagi kasus berdasarkan $n$ modulo $4$.

\begin{enumerate}
    \item Jika $n=4k\geq 16$, 
    kita dapat langsung menggunakan klaim untuk mendapatkan
    \[f(n) \leq |k-4| = \dfrac{n-16}{4} < \dfrac{n-5}{4}.\]
    
    \item Jika $n=4k+2 \geq 18$,
    pertama-tama kita tambahkan atau kurangkan $2$ untuk mendapatkan $4k$ atau $4k+4$. Kemudian kita terapkan klaim tersebut. Dengan mengikuti strategi ini, kita dapat
    \[f(n) \ leq 1+ |(k+1)-4| = \dfrac{n-10}{4} < \dfrac{n-5}{4}.\]
    
    \item Jika $n=4k+3 \geq 19$ dan langkah pertama Neneng adalah "naik", Asep dapat mengubah bilangan tersebut menjadi $4k+4$. Maka,
    \[f(n) \leq 1+|(k+1)-4| =\dfrac{n-11}{4} < \dfrac{n-5}{4}.\]
    
    \item Jika $n=4k+3 \geq 19$ dan langkah pertama Neneng adalah "turun", Asep dapat mengubah bilangan tersebut menjadi $4k+2$. Maka,
    \[f(n) \leq 1+\dfrac{(n-1)-10}{4} =\dfrac{n-7}{4} < \dfrac{n-5}{4}.\]
    
    \item Jika $n=4k+1 \geq 17$ dan langkah pertama Neneng adalah "turun", 
    Asep dapat mengubah bilangan tersebut menjadi $4k$. Maka,
    \[f(n) \leq 1+|k-4| =\dfrac{n-13}{4} < \dfrac{n-5}{4}.\]
    
    \item Jika $n=4k+1 \geq 17$ dan langkah pertama Neneng adalah "naik", 
    Asep dapat mengubah bilangan tersebut menjadi $4k+2$. Maka,
    \[f(n) \leq 1+\dfrac{(n+1)-10}{4} =\dfrac{n-5}{4}.\]
\end{enumerate}

Sisa kasus lainnya adalah $n=14$, di mana Asep dapat mengganti $14 \to 16$ atau $14 \to 12 \to (9 \text{ atau } 16)$, dan $n=15$, di mana Asep dapat mengganti $15 \to 16$ atau $15 \to 12 \to (9 \text{ atau } 16)$. Terbukti.
\end{solusi}
\subsection{Nomor 4 Hari Pertama}
Tentukan ada atau tidak bilangan asli $N$ yang memenuhi tiga syarat berikut:
    \begin{itemize}
        \item $N$ habis dibagi $2^{2023}$, tapi tidak habis dibagi $2^{2024}$,
        \item $N$ hanya memuat tiga digit berbeda, dan $N$ tidak memiliki digit nol,
        \item Tepat $99.9\%$ digit $N$ merupakan bilangan ganjil.
    \end{itemize}
\textit{(Soal di\textit{propose} oleh Muhammad Afifurrahman (Emas OSN 2015))}
\begin{motivasi}(oleh Muhammad Afifurrahman)
    Inti utama dari masalah ini (ketika saya memikirkan soal ini) adalah klaim dalam solusi tersebut. Bagian yang paling sulit setelah membuktikan klaim-klaim tersebut adalah untuk menyembunyikan klaim tersebut dalam soal dan pastinya $99.9 \%$ itu cuma jadi pengalih perhatian; setiap pecahan pada interval $(0,1)$ harusnya bisa (dan mungkin juga untuk nol?). Selain itu, dengan menambahkan beberapa digit di depan, ada tak terhingga banyaknya $N$ yang memenuhi pernyataan tersebut.
\end{motivasi}

\begin{solusi}(oleh Muhammad Afifurrahman)
    Pertama akan dibuktikan klaim yang lebih kuat.
    
    \begin{claim*}
        Terdapat barisan bilangan bulat positif $(a_i)_{i=3}^\infty$ sehingga pernyataan berikut berlaku: 
        \begin{itemize}
            \item Untuk semua $n\geq 3$, $a_n$ hanya mengandung digit $1, 2, 3,$
            
            \item $a_n$ memiliki tepat $n$ digit,
            
            \item $v_2(a_n)=n$, dan 
            
            \item semua digit pada $a_n$, kecuali digit terakhir, bernilai ganjil.
        \end{itemize}
        \begin{bukti}
            Akan dilakukan induksi pada $n$. 
            
            Untuk $n=3$, ambil $a_3=312$. Misalkan ada $a_k$ yang memenuhi untuk suatu $k\geq 3$. Ternyata, baik $10^k+a_k$ maupun $3 \cdot 10^k+a_k$ dapat dibagi oleh $2^{k+1}$, dikarenakan
            
            \[ 10^k+a_k \equiv 2^k + 2^k \equiv 0 \pmod {2^{k+1}}, \]
            
            dan salah satunya tidak dapat dibagi oleh $2^{k+2}$ (karena selisih kedua bilangan tersebut adalah $2 \cdot 10^k$ yang tidak dapat dibagi oleh $2^{k+2}$). Oleh karena itu, dapat dipilih $a_{k+1}$ sebagai salah satu dari dua bilangan tersebut. Klaim terbukti,
        \end{bukti}
    \end{claim*}

    Dari klaim tersebut, maka kita dapat mengkonstruksi $N$ dengan $3000$ digit yang memenuhi syarat di soal. Ambil $2023$ digit terakhir dari $N$ sebagai $a_{2023}$ yang telah dikonstruksi sebelumnya, dan ambil 977 digit pertama dari $N$ sebagai $3 \ldots 322$, dengan 975 digit 3 dan dua digit 2. Dengan konstruksi $a_n$, dapat dilihat bahwa $N$ memiliki tiga digit genap yang semuanya adalah 2, dan 2997 digit ganjil yang merupakan 1 atau 3. Oleh karena itu, konstruksi ini memenuhi persyaratan persentase dan digit. 
    Selanjutnya, kita punya
    \[ N = 3\ldots 322 \cdot 10^{2023} + a_{2023}. \]
    Suku pertama dapat dibagi oleh $2^{2024}$, dan suku kedua dapat dibagi oleh $2^{2023}$, tetapi tidak dapat dibagi oleh $2^{2024}$. Oleh karena itu, $v_2(N)=2023$. Konstruksi tersebut memenuhi. Dapat disimpulkan bahwa bilangan $N$ yang dimaksud ada. Terbukti
\end{solusi}
\subsection{Nomor 8 Hari Kedua}
Diberikan tiga bilangan bulat positif berbeda $a, b, c$. Definisikan $S(a, b, c)$ sebagai himpunan semua akar rasional dari $px^2 + qx + r = 0$ untuk semua $(p, q, r)$ yang merupakan permutasi dari $(a, b, c)$. Sebagai contoh, $(1, 2, 3) = \{-1,-2, -\frac{1}{2}\}$ karena persamaan $x^2 + 3x + 2 = 0$ memiliki akar $-1$ dan $-2$, persamaan $2x^2 + 3x + 1 = 0$ memiliki akar $-1$ dan $-\frac{1}{2}$, sedangkan untuk permutasi yang lainnya persamaan kuadrat yang terbentuk tidak memiliki akar rasional. Tentukan maksimum banyaknya elemen di $S(a, b, c)$.
\newline
\textit{(Soal di\textit{propose} oleh Valentio Iverson (Absolute Winner, Emas OSN 2019))}
\begin{motivasi}
    (oleh Valentio Iverson) Kesulitan utama dari soal ini terletak pada konstruksi, yang menurut saya sebenarnya cukup masuk akal. 
    
    Kita bisa mulai dengan mengerjakan kasus-kasus untuk $\min(a,b,c)$, sehingga dapat dibuktikan bahwa $\min(a,b,c) = 1$ tidak berhasil.
    
    Kemudian, coba $\min(a,b,c) = 2$, atau ketika kita menginginkan agar $b^2 - 8c$ dan $c^2 - 8b$ adalah kuadrat sempurna. Intuisi yang paling "natural" untuk dicoba adalah memaksa $b$ menjadi polinomial dalam $c$ sehingga $b^2 - 8c$ adalah kuadrat dari polinomial dalam $c$, yang berarti kita ingin sesuatu yang berbentuk seperti $b = 2c + 1$. Kemudian sisanya gampang, coba paksa agar $c^2 - 8(2c + 1) = (c - 8)^2 - 72$ menjadi kuadrat sempurna.
\end{motivasi}
\begin{solusi}(oleh Valentio Iverson)
    Jawabannya adalah $\boxed{8}$, yang tercapai saat $(a,b,c) = (2,17,35)$. Dapat dilihat bahwa pada konstruksi tersebut,
    \[ S(a,b,c) = \left \{-17, - 5,  -\frac{7}{2},  -2, - \frac{1}{2},  - \frac{1}{5}, - \frac{2}{7}, - \frac{1}{17}  \right \}. \]
    
    Untuk membuktikan bahwa batasan tersebut adalah yang paling besar, dapat dilihat agar $px^2 + qx + r$ memiliki solusi rasional, maka haruslah $q^2 - 4pr \ge 0$.
    
    Perhatikan bahwa ada permutasi $(d,e,f)$ dari $(a,b,c)$ sehingga $d < e < f$. 
    
    Dalam kasus ini, $d^2 - 4ef < 0$ dan oleh karena itu baik $ex^2 + dx + f$ maupun $fx^2 + dx + e$ tidak memberikan akar yang rasional. 
    
    Oleh karena itu, paling banyak hanya ada $2 \cdot 4 = 8$ akar rasional dari $4$ persamaan kuadrat yang tersisa, seperti yang ingin dibuktikan.
\end{solusi}

\end{document}
